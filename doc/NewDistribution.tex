\documentclass[pdftex,titlepage,10pt]{article}

\usepackage{a4wide}
\usepackage{times}
\usepackage{german}
\usepackage[T1]{fontenc}
\usepackage[ansinew]{inputenc}
\usepackage{makeidx}
\usepackage[final]{graphicx}
%\usepackage[draft]{graphicx}

\usepackage[colorlinks,bookmarks,pdfauthor={%
  Michael Hager, Michael K\"ohrmann}%
,pdfsubject={RC-Config-Editor}%
,pdftitle={RC-Config-Editor}%
,pdfkeywords={}]{hyperref}

%\makeindex
%%%
\newcommand{\HRule}{\rule{\linewidth}{1mm}}
\title{%
  \HRule\\[0.7cm]
  RC-Config-Editor\\[0.5em]
  Anpassen der Metadaten bei Erscheinen einer neuen SuSE-Distribution\\[0.5cm]
  \HRule\\\vfill}
\author{Michael Hager, {\tt mike@suse.de}\\
  Michael K\"ohrmann, {\tt curry@suse.de}\\
  SuSE GmbH N\"urnberg}
\date{\today}
%%%
\begin{document}
\setlength{\parindent}{0pt}
\pdfcompresslevel=9
\DeclareGraphicsExtensions{.pdf,.jpg,.png}
\pagestyle{headings}
%%%%%
\maketitle
%\tableofcontents
%\newpage
%%%%%
\section{Was leistet diese Anleitung}
Dieses Dokument soll eine Arbeitsanleitung geben, um die Metadaten des
{\em RC-Config-Editors} an eine neue {\em SuSE}-Distribution
anzupassen, d.h. eine neue und vervollst\"andigte {\tt meta\_rc.config}
zu erzeugen. Die Metadaten der {\tt rc.config}-Variablen sind
zus\"atzliche beschreibende Daten, wie z.B. den Datentyp der Variablen
({\tt boolean}, {\tt integer}, \ldots).\\

\section{Allgemeine Einf\"uhrung}
Zum {\em YaST2}-Modul des {\em RC-Config-Editors} geh\"ort eine
Datenbank, in der Metadaten gespeichert sind. Die Metadaten speichern
Informationen \"uber die Konfigurationsvariablen.
Auf der Grundlage dieser Daten und der Inhalte der Dateien {\tt
  /etc/rc.config}, {\tt /etc/rc.config.d/*.rc.config} und {\tt
  /etc/rc.dialout} wird der Baum generiert, mit dem man im 
{\em RC-Config-Editor} durch die Variablen und Variablengruppen
navigieren kann. Die Datenbank speichert die Zuordnung der Variablen
in die einzelnen Gruppen, sowie die Datentypen und Typdefinitionen der
Variablen und Beschreibungen der Gruppenverzeichnisse.\\

Die Zuordnung neuer Variablen in den Gruppenbaum kann nicht
automatisch erfolgen.Einige Softwarepakete ben\"otigen bestimmte Variablen,
die w\"ahrend der Installation dieser Pakete in den Konfigurationsdateien
gespeichert werden. Bei Neuerscheinen einer {\em
  SuSE}-Distribution k\"onnen also unter Umst\"anden neue Variablen in die
Konfigurationsdateien eingetragen werden.\\

Die Metadaten werden in der Datei {\tt /usr/lib/YaST2/meta\_rc.config}
abgelegt.
%%%
\newpage
\section{Erstellen einer neuen {\tt meta\_rc.config}}
%%%
\subsection*{1. Schritt: Zugriff auf die aktuelle Version des
  {\em RC-Config-Editors} und seiner Quelldateien}
\begin{description}
\item[Erkl\"arung:] Zun\"achst mu\ss \ man sich Zugriff auf die
  aktuelle {\em YaST2-Version} und alle zugeh\"origen Werkzeuge
  verschaffen.
\item[Todo:] {\tt cvs}-Zugriff verschaffen.
\end{description}
\begin{description}
\item[Erkl\"arung:] Update der aktuellen {\tt y2c\_rc\_config}-Version
  mit {\tt cvs}.
\item[Todo:]
{\footnotesize \begin{verbatim}
# ~/ > cd /local/yast2/modules/y2c_rc_config/
# /local/yast2/modules/y2c_rc_config > cvs up
\end{verbatim}}
\item[Ergebnis:] Die aktuellen Quelldateien des {\em
    RC-Config-Editors}.
\end{description}
%%%
\subsection*{2. Schritt: Erstellen eines neuen Verzeichnisbaumes}
\begin{description}
\item[Erkl\"arung:] Um neue Variablen einzuordnen, erzeugt
  man sich einen neuen Verzeichnisbaum aus der aktuellen
  {\tt meta\_rc.config}-Datei. Dazu kopiert man die Datei
  {\tt meta\_rc.config} und das Perl-Skript
  {\tt create\_dirtree\_from\_eddb.pl}. Das Ausf\"uhren des
  Skripts auf Basis der neuen Meta-Datenbank erzeugt ein neues Verzeichnis
  {\tt root/} mit allen Gruppen und bisherigen Variablen.
\item[Todo:]
{\footnotesize \begin{verbatim}
# ~/ > cp /local/yast2/modules/y2c_rc_config/src/meta_rc.config .
# ~/ > cp /local/yast2/modules/y2c_rc_config/src/create_dirtree_from_eddb.pl .
# ~/ > ./create_dirtree_from_eddb.pl ./meta_rc.config
Reading meta_rc.config...
Processing directories and array...
Reading meta_rc.config...
Processing type, mtype, typedef and descr files...
# ~/ > ls root/
Base-Administration  Desktop   Hardware  Network           Security
Base-Applications    Firewall  Mail      Network-Services  Start-Variables
\end{verbatim}}
\item[Ergebnis:] Man erh\"alt einen aktuellen Verzeichnisbaum in seinem
  Heimatverzeichnis unter dem Pfad \verb+$HOME/root/+.
\end{description}
%%%$
\subsection*{3. Schritt: Installieren aller Konfigurationsvariablen}
\begin{description}
\item[Erkl\"arung:] Durchf\"uhren einer Maximalinstallation der neuen
  {\em SuSE}-Distribution. Durch die Installation aller verf\"ugbarer
  Pakete werden (fast) alle m\"oglichen Konfigurationsvariablen in den
  Dateien unter {\tt /etc/rc.config}, {\tt /etc/rc.dialout} und {\tt
    /etc/rc.config.d/*.rc.config} angelegt.
\item[Todo:] F\"ur jede der Installationsmedien (hier CD-Rom)f\"uhrt man
die folgenden Schritte durch: {\footnotesize
\begin{verbatim}
# ~/ > mount /cdrom/
# ~/ > cd /cdrom/suse/ 
# /cdrom/suse > find -name "*.rpm" -exec rpm -hUv --nodeps --force {} \;
# /cdrom/suse > cd # ~/ > umount /cdrom/
\end{verbatim}}
  \item[Hinweis:] Die Installation aller Pakete sollte auf einem
    Testrechner durchgef\"uhrt werden, da das System danach evtl.
    nicht mehr vollst\"andig lauff\"ahig ist.
  \item[Ergebnis:] In den Konfigurationsdateien werden alle m\"oglichen
    Variablen angelegt.
\end{description}
%%%
\subsection*{4. Schritt: Finden der Variablen, \"uber die noch keine Metadaten
  vorliegen}
\begin{description}
\item[Erkl\"arung:] Mit Hilfe eines Skriptes k\"onnen die Variablen
  leicht herausgefunden werden, die noch nicht in der Datenbank der
  Metadaten aufgenommen sind.
\item[Todo:]
{\footnotesize\begin{verbatim}
# ~/ > /usr/lib/YaST2/bin/get_unknown_variables.sh
creating /tmp/rc.config.all
creating /tmp/rc.config.variables
creating /tmp/meta_rc.config.variables
creating /tmp/variables.diff
creating /tmp/unknown_variables.txt
\end{verbatim}}
\item[Ergebnis:] Man erh\"alt die Datei {\tt
    /tmp/unknown\_variables.txt}, in der alle Variablen enthalten
  sind, die noch nicht in der Metadatenbank aufgenommen sind.
\item[Hinweis:] Am Ende dieser Datei sind evtl. einige Eintr\"age
  vorhanden, die jedoch keine g\"ultigen Variablen
  darstellen, wie z.B.  {\tt test}, {\tt true} oder {\tt umask}.
  Diese Eintr\"age sind in den folgenden Arbeitsschritten unwichtig
  und k\"onnen entfernt werden.
\end{description}
%%%
\subsection*{5. Schritt: Zuordnung der gefundenen Variablen}
\begin{description}\label{SchrittAnfang}
\item[Erkl\"arung:] Nun m\"ussen alle gefundenen Variablen aus der Datei
  {\tt /tmp/unknown\_variables.txt} im Variablenbaum der richtigen
  Gruppe zugeordnet werden. Bei dieser Zuordnung sollte man sich an der
  bisherigen Ordnung des Baumes orientieren und m\"oglichst die
  Variablen in Gruppen unterbringen, die bereits im Baum vorhanden sind.
  Evtl. ist es aber sinnvoll, eine neue Gruppe anzulegen (siehe dazu
  \ref{newgroup}). Um die richtige Zuordnung einer Variable herauszufinden,
  kann man deren Beschreibung in den Konfigurationsdateien (z.B. in
  {\tt /tmp/rc.config.all}) und {\tt man}- oder {\tt info}-Seiten heranziehen.
\item[Todo:] Finden des Pfades der thematisch richtigen Gruppe zur 
  jeweiligen Variable.
\item[Beispiel:] Variable: {\tt I4L\_START}, zugeh\"orige Gruppe:
  {\tt Start-Variables/Start-Network}
{\footnotesize\begin{verbatim}
# ~/ > cd rc_
\end{verbatim}}
\end{description}
%%%
\subsection*{6. Schritt: Erstellen einer neuen Gruppe im Verzeichnisbaum}\label{newgroup}
\begin{description}
\item[Erkl\"arung:] Kann man eine neue Variable keiner der bisherigen Gruppen
  im Verzeichnisbaum zuordnen, so mu\ss \ man eine neue Gruppe anlegen. Eine
  Gruppe umfa\ss t immer ein Verzeichnis mit dem jeweiligen Gruppennamen, einer
  Datei {\tt descr}, die im Gruppenverzeichnis abgelegt ist und dieser Gruppe
  zugeordnete Konfigurationsvariablen.
\item[Todo:]
{\footnotesize\begin{verbatim}
# ~/rc_config_tree_copy/root > mdkir Start-Variables/Start-Network
# ~/rc_config_tree_copy/root > cat > Start-Variables/Start-Network/descr
<p></p>
<p>Start variables for network service deamons like Apache and Samba</p>
(Ctrl-D)
# ~/rc_config_tree_copy/root >
\end{verbatim}}
\end{description}
%%%
\subsection*{7. Schritt: Ablegen der Variable im Verzeichnisbaum}
\begin{description}
\item[Erkl\"arung:] F\"ur jede Konfigurationsvariable mu\ss \ im Verzeichnisbaum
  ein neues Unterverzeichnis mit dem Namen der jeweiligen Variable angelegt werden.
\item[Todo:] Beispiel {\tt I4L\_START}:
{\footnotesize\begin{verbatim}
# ~/ > cd rc_config_tree_copy/root/
# ~/rc_config_tree_copy/root > mdkir Start-Variables/Start-Network/I4L_START
\end{verbatim}}
\end{description}
%%%
\subsection*{8. Schritt: Herausfinden des Datentyps einer Variable}
\begin{description}
\item[Erkl\"arung:] Durch die Angabe des Datentyps einer Variable kann die korrekte
  Zuweisung von Werten an die Variable vereinfacht werden. Der {\em RC-Config-Editor}
  kennt vier verschiedene Datentypen von Konfigurationsvariablen:
  \begin{enumerate}
  \item {\tt boolean},
  \item {\tt integer},
  \item \verb+enum "..."+ und
  \item {\tt string}.
  \end{enumerate}
\item[Todo:] Mit Hilfe der Beschreibungen der Variablen in den Konfigurationsdateien
  {\tt /etc/rc.config} bzw. {\tt /etc/rc.config.d/*.rc.config} soll der Datentyp
  der Variablen m\"oglichst streng eingegrenzt werden (siehe dazu auch die Tabelle
  auf S. \pageref{tabelle}).
\end{description}
%%%
\subsection*{9. Schritt: Zuordnung des Datentyps zu einer Variable}
\begin{description}
\item[Erkl\"arung:] Nun mu\ss \ die Information \"uber den Datentyp der Variable in
  den Verzeichnisbaum eingetragen werden, um letztendlich in den Metadaten abgespeichert 
  zu werden.
\item[Todo:] Beispiel {\tt I4L\_START}:
{\footnotesize\begin{verbatim}
# ~/rc_config_tree_copy/root > cat > Start-Variables/Start-Network/I4L_START/type
boolean
(Ctrl-D)
# ~/rc_config_tree_copy/root >
\end{verbatim}}
\end{description}
%%%
\subsection*{10. Schritt: Herausfinden der Typdefinition einer Variable}
\begin{description}
\item[Erkl\"arung:] Neben dem Datentyp einer Variablen kann man Angaben \"uber
  die Typdefinition machen. Diese macht Aussagen \"uber die Notwendigkeit einer
  strikten bzw. nicht strikten Einhaltung des Datentypen. Siehe dazu auch die
  Tabelle auf S. \pageref{tabelle}.
\item[Todo:] Mit Hilfe der Beschreibungen der Variablen in den Konfigurationsdateien
  {\tt /etc/rc.config} bzw. {\tt /etc/rc.config.d/*.rc.config} soll der Datentyp
  der Variablen m\"oglichst streng eingegrenzt werden.
\end{description}
%%%
\subsection*{11. Schritt: Zuweisung der Typdefinition zu einer Konfigurationsvariable}
\begin{description}\label{SchrittEnde}
\item[Erkl\"arung:] Damit die Information \"ueber die Typdefinition in
  die Datei {\tt meta\_rc.config} abgespeichert werden kann, mu\ss \ im 
  Unterverzeichnis der jeweiligen Variable eine Datei {\tt typedef} angelegt
  werden, in der entweder {\tt strict} f\"ur die strikte Einhaltung des Datentypen
  eingetragen ist, oder {\tt not\_strict} f\"ur den umgekehrten Fall.
\item[Todo:] Beispiel {\tt I4L\_START}:
{\footnotesize\begin{verbatim}
# ~/rc_config_tree_copy/root > cat > Start-Variables/Start-Network/I4L_START/typedef
strict
(Ctrl-D)
# ~/rc_config_tree_copy/root >
\end{verbatim}}
\end{description}
%%%
\subsection*{12. Schritt: Erstellen der Metadatenbank {\tt EDDB}}
\begin{description}
\item[Erkl\"arung:] Nachdem man die Schritte \ref{SchrittAnfang} bis
  \ref{SchrittEnde} f\"ur jede der neuen Variablen durchgef\"uhrt hat, mu\ss \ die
  neue Metadatenbank erstellt werden. Dazu verwendet man das {\em Perl}-Skript
  {\tt create\_eddb.pl}.
\item[Todo:]
{\footnotesize\begin{verbatim}
# ~/rc_config_tree_copy/root > create_eddb.pl
Creating EDDB.tmp...
Sorting...
The new database file ../EDDB was successfully created.
# ~/rc_config_tree_copy/root > cd ../
# ~/rc_config_tree_copy > ls -l EDDB
-rw-r--r--    1 user    suse        29242 Jan 29 14:20 EDDB
# ~/rc_config_tree_copy > 
\end{verbatim}}
\item[Ergebnis:] Man erh\"alt die vollst\"andige Metadatenbank {\tt EDDB}.
\end{description}
%%%
\subsection*{13. Schritt: Kopieren der neuen Metadatenbank in das Quellenverzeichnis}
\begin{description}
\item[Erkl\"arung:] Die neu erstellte Metadatenbank {\tt EDDB} mu\ss \ noch
  in den {\em CVS}-Baum eingebunden werden.
\item[Todo:]
\item[Ergebnis:]
\end{description}
%%%
%\subsection{}
%\begin{description}
%\item[Erkl\"arung:]
%\item[Todo:]
%\item[Ergebnis:]
%\end{description}

\begin{table}
\begin{center}\label{tabelle}
  \begin{tabular}{|l|p{5.5cm}|p{5.5cm}|}
    \hline
    & {\tt strict} & {\tt not\_strict}\\
    \hline
    {\tt boolean} & Es k\"onnen nur die Werte {\tt yes} oder {\tt no}
    angenommen werden. & Neben {\tt yes} oder {\tt no} kann auch ein
    beliebiger anderer Wert angegeben werden. (Sinnvoller w\"are
    vielleicht {\tt enum}).\\[.5em]
    {\tt integer} & Es k\"onnen nur ganzzahlige Werte angenommen werden
    (wird jedoch noch nicht vom Editor \"uberpr\"uft. & Neben ganzzahligen
    Werten k\"onnen auch andere beliebige Werte angegeben werden, bspw.
    Strings.\\[.5em]
    {\tt enum }   & Es k\"onnen nur die Werte aus der angegebenen Liste
    angenommen werden. & Neben den Werten aus der angegebenen Liste
    k\"onnen auch noch zus\"atzliche angegeben werden.\\[.5em]
    {\tt string}  & Es kann ein beliebiger String angegeben werden. &
    Es kann ein beliebiger String angegeben werden, genauso wie bei
    {\tt strict}.\\
    \hline
  \end{tabular}
\caption{Kombinationen von Datentyp und Typdefinition bei Konfigurationsvariablen}
\end{center}
\end{table}
%%%
\section{Beispiele}
{\footnotesize \begin{verbatim}
CHECK_PERMISSIONS    path       /Base-Administration/SuSEConfig
CHECK_PERMISSIONS    type       enum "set,warn,no"
CHECK_PERMISSIONS    typedef    strict
DHCRELAY_SERVERS     mtype      enum "127.0.0.1,127.0.0.2"
DHCRELAY_SERVERS     path       /Network/Dhcp
DHCRELAY_SERVERS     typedef    not_strict
\end{verbatim}}

{\footnotesize \begin{verbatim}
# ~/rc_config_tree/root/Base-Administration/SuSEConfig/CHECK_PERMISSIONS > ls -l
insgesamt 8
-rwxrwxrwx    1 mike    suse           19 Jan 16 08:59 type
-rwxrwxrwx    1 mike    suse            7 Jan 16 08:59 typedef
# ~/rc_config_tree/root/Base-Administration/SuSEConfig/CHECK_PERMISSIONS > cat type
enum "set,warn,no"
# ~/rc_config_tree/root/Base-Administration/SuSEConfig/CHECK_PERMISSIONS > cat type
strict
\end{verbatim}}
%%%
\end{document}

%%% Local Variables: 
%%% mode: latex
%%% TeX-master: "Todo"
%%% End: 
